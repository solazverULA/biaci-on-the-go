%% Generated by Sphinx.
\def\sphinxdocclass{report}
\documentclass[letterpaper,10pt,spanish]{sphinxmanual}
\ifdefined\pdfpxdimen
   \let\sphinxpxdimen\pdfpxdimen\else\newdimen\sphinxpxdimen
\fi \sphinxpxdimen=.75bp\relax

\PassOptionsToPackage{warn}{textcomp}
\usepackage[utf8]{inputenc}
\ifdefined\DeclareUnicodeCharacter
 \ifdefined\DeclareUnicodeCharacterAsOptional
  \DeclareUnicodeCharacter{"00A0}{\nobreakspace}
  \DeclareUnicodeCharacter{"2500}{\sphinxunichar{2500}}
  \DeclareUnicodeCharacter{"2502}{\sphinxunichar{2502}}
  \DeclareUnicodeCharacter{"2514}{\sphinxunichar{2514}}
  \DeclareUnicodeCharacter{"251C}{\sphinxunichar{251C}}
  \DeclareUnicodeCharacter{"2572}{\textbackslash}
 \else
  \DeclareUnicodeCharacter{00A0}{\nobreakspace}
  \DeclareUnicodeCharacter{2500}{\sphinxunichar{2500}}
  \DeclareUnicodeCharacter{2502}{\sphinxunichar{2502}}
  \DeclareUnicodeCharacter{2514}{\sphinxunichar{2514}}
  \DeclareUnicodeCharacter{251C}{\sphinxunichar{251C}}
  \DeclareUnicodeCharacter{2572}{\textbackslash}
 \fi
\fi
\usepackage{cmap}
\usepackage[T1]{fontenc}
\usepackage{amsmath,amssymb,amstext}
\usepackage{babel}
\usepackage{times}
\usepackage[Sonny]{fncychap}
\ChNameVar{\Large\normalfont\sffamily}
\ChTitleVar{\Large\normalfont\sffamily}
\usepackage{sphinx}

\usepackage{geometry}

% Include hyperref last.
\usepackage{hyperref}
% Fix anchor placement for figures with captions.
\usepackage{hypcap}% it must be loaded after hyperref.
% Set up styles of URL: it should be placed after hyperref.
\urlstyle{same}
\addto\captionsspanish{\renewcommand{\contentsname}{Contents:}}

\addto\captionsspanish{\renewcommand{\figurename}{Figura}}
\addto\captionsspanish{\renewcommand{\tablename}{Tabla}}
\addto\captionsspanish{\renewcommand{\literalblockname}{Lista}}

\addto\captionsspanish{\renewcommand{\literalblockcontinuedname}{proviene de la página anterior}}
\addto\captionsspanish{\renewcommand{\literalblockcontinuesname}{continues on next page}}

\addto\extrasspanish{\def\pageautorefname{página}}

\setcounter{tocdepth}{1}



\title{Biaci On The Go Documentation}
\date{25 de junio de 2018}
\release{0.1}
\author{Miguel Méndez, Yenifer Ramírez y Manuel Zambrano}
\newcommand{\sphinxlogo}{\vbox{}}
\renewcommand{\releasename}{Versión}
\makeindex

\begin{document}
\ifnum\catcode`\"=\active\shorthandoff{"}\fi
\maketitle
\sphinxtableofcontents
\phantomsection\label{\detokenize{index::doc}}



\chapter{Indices and tables}
\label{\detokenize{index:indices-and-tables}}\begin{itemize}
\item {} 
\DUrole{xref,std,std-ref}{genindex}

\item {} 
\DUrole{xref,std,std-ref}{modindex}

\item {} 
\DUrole{xref,std,std-ref}{search}

\end{itemize}


\chapter{Contents:}
\label{\detokenize{index:contents}}

\section{Views Aplicacion de Reserva}
\label{\detokenize{modules/reserva/views:module-reserva.views}}\label{\detokenize{modules/reserva/views:views-aplicacion-de-reserva}}\label{\detokenize{modules/reserva/views::doc}}\index{reserva.views (módulo)}\index{ErrorReserva (clase en reserva.views)}

\begin{fulllineitems}
\phantomsection\label{\detokenize{modules/reserva/views:reserva.views.ErrorReserva}}\pysiglinewithargsret{\sphinxbfcode{\sphinxupquote{class }}\sphinxcode{\sphinxupquote{reserva.views.}}\sphinxbfcode{\sphinxupquote{ErrorReserva}}}{\emph{**kwargs}}{}
Esta clase muestra un error si la reserva no se pudo realizar

\end{fulllineitems}

\index{ReservaLibros (clase en reserva.views)}

\begin{fulllineitems}
\phantomsection\label{\detokenize{modules/reserva/views:reserva.views.ReservaLibros}}\pysiglinewithargsret{\sphinxbfcode{\sphinxupquote{class }}\sphinxcode{\sphinxupquote{reserva.views.}}\sphinxbfcode{\sphinxupquote{ReservaLibros}}}{\emph{**kwargs}}{}
Esta clase verifica las reservas realizadas por el usuario
\index{get() (método de reserva.views.ReservaLibros)}

\begin{fulllineitems}
\phantomsection\label{\detokenize{modules/reserva/views:reserva.views.ReservaLibros.get}}\pysiglinewithargsret{\sphinxbfcode{\sphinxupquote{get}}}{\emph{request}, \emph{*args}, \emph{**kwargs}}{}
Funcion para recopilar la informacion de los ejemplares ejemplares reservado por el usuario logueado

A partir de los datos de la consulta, se listan los datos de la reserva
\begin{quote}\begin{description}
\item[{Devuelve}] \leavevmode
Una plantilla llamada \sphinxtitleref{lista\_reserva.html}

\item[{Tipo del valor devuelto}] \leavevmode
Template Response

\end{description}\end{quote}

\end{fulllineitems}


\end{fulllineitems}

\index{Reservar() (en el módulo reserva.views)}

\begin{fulllineitems}
\phantomsection\label{\detokenize{modules/reserva/views:reserva.views.Reservar}}\pysiglinewithargsret{\sphinxcode{\sphinxupquote{reserva.views.}}\sphinxbfcode{\sphinxupquote{Reservar}}}{\emph{request}, \emph{id\_ejemplar}}{}
Funcion para reservar un ejemplar.

0- Verifica que el mismo ejemplar no haya sido reservado en los ultimos dos dias por el mismo usuario
1- Reserva el ejemplar.
2. Guarda la reserva en un historial.
3. Cambia el estado del ejemplar de disponible a reservado.
\begin{quote}\begin{description}
\item[{Devuelve}] \leavevmode
un render a \sphinxtitleref{reservar.html}

\item[{Tipo del valor devuelto}] \leavevmode
render

\end{description}\end{quote}

\end{fulllineitems}



\renewcommand{\indexname}{Índice de Módulos Python}
\begin{sphinxtheindex}
\def\bigletter#1{{\Large\sffamily#1}\nopagebreak\vspace{1mm}}
\bigletter{r}
\item {\sphinxstyleindexentry{reserva.views}}\sphinxstyleindexpageref{modules/reserva/views:\detokenize{module-reserva.views}}
\end{sphinxtheindex}

\renewcommand{\indexname}{Índice}
\printindex
\end{document}